\documentclass[12pt,a4paper,oneside]{ctexbook}
\usepackage{amsmath, amsthm, amssymb, bm, graphicx, hyperref, mathrsfs}

\title{{\Huge{\textbf{解析几何教程(第三版)廖华奎\\ 全书总结}}}\\(不完美总结)}
\author{Buce-Ithon}
\date{\today}
\linespread{1.5}

\newtheorem{theorem}{定理}[section]
\newtheorem{definition}[theorem]{定义}
\newtheorem{lemma}[theorem]{引理}
\newtheorem{corollary}[theorem]{推论}
\newtheorem{example}[theorem]{例}
\newtheorem{proposition}[theorem]{命题}


\begin{document}

\maketitle

\pagenumbering{roman}
\setcounter{page}{1}
\begin{center}
	\Huge\textbf{前言}
\end{center}~\


首先,十分感谢感谢读者(My Dear Friend!)您打开这份还没有尘封的笔记,您能抽出一些时间阅读鄙人所写的一些总结是我莫大的荣幸!


如您所见,这是一份解析几何教程(第三版)(廖华奎 王宝富 编著)的全书总结,也是笔者本科一年级开设的一门课程。这份总结写于课程考试之后,旨在帮助有兴趣阅读这本书的同学快速了解或者总结全书内容。并且本书难度不大,读者仅需掌握部分线性代数的知识即可放心阅读(当然,遇到相关的知识我也会在相应章节的结尾给出注释),所以作为闲暇之余的休闲文章阅读也是非常适合的。


当然,鉴于笔者水平所限,笔记中难免会有所纰漏,希望读者可以在阅读过程中理性思考,在下虚心接受一切理性的批评指正等反馈。
~\\
\begin{flushright}
	\begin{tabular}{c}
		Buce-Ithon\\
		email:2723896502@qq.ocm\\
		\today
	\end{tabular}
\end{flushright}

\newpage
\pagenumbering{Roman}
\setcounter{page}{1}
\tableofcontents
\newpage
\setcounter{page}{1}
\pagenumbering{arabic}

\chapter{向量代数}

本章主要介绍坐标与标架以及向量的三种运算:内积、外积、混合积(大部分都是中学学过的内容)

\section{向量及其线性运算}

向量(或矢量):既有大小又有方向的量,可用符号\textbf{a,b,c,...}表示,或者用有向线段表示(诸如:)

\section{标架与坐标}

\chapter{直线与平面}

此处可以输入笔记内容

\section{}

这是笔记的正文部分

\chapter{常见曲面}

此处可以输入笔记内容

\section{}

这是笔记的正文部分

\chapter{二次曲线和二次曲线}

此处可以输入笔记内容

\section{}

这是笔记的正文部分

\chapter{正交变换和仿射变换}

此处可以输入笔记内容

\section{}

这是笔记的正文部分

\chapter{平面射影几何简介}

此处可以输入笔记内容

\section{}

这是笔记的正文部分

\chapter{球面几何与双曲几何初步}

此处可以输入笔记内容

\section{}

这是笔记的正文部分
\end{document}
